\section{Требования к сдаче-приемке проекта}
\subsection{Требования к срокам выполнения проекта}

\subsection{Требования к наполнению информацией}
Общие требования к информационному наполнению.

\subsection{Требования к документации}
В момент сдачи проекта заказчику предоставляется следующий набор документов:
\begin{enumerate}
	\item Техническое задание;
	\item Краткое руководство пользователя по работе в административной панели разрабатываемых разделов;
	\item Файл с данными для входа в административную панель сайта/хостинга;
	\item Графические материалы;
	\item План и сроки;
	\item Документ с архитектурой проекта.
\end{enumerate}

\subsection{Требования к персоналу}
Для работы в административной панели управления разрабатываемых разделов от администратора не должно требоваться специальных технических навыков, знания технологий или программных продуктов, за исключением общих навыков работы с персональным компьютером и стандартными веб-браузерами (например, Chrome, Safari, Firefox и др.).

\subsection{Порядок предоставления дистрибутива}
По окончании разработки Исполнитель должен предоставить Заказчику дистрибутив системы в следующем виде:
\begin{enumerate}
	\item архив с исходным кодом всех программных модулей разрабатываемых разделов;
	\item дамп проектной базы данных с актуальной информацией;
\end{enumerate}

Дистрибутив предоставляется в виде файлового архива.

\subsection{Порядок переноса сайта на технические средства заказчика}

После завершения сдачи-приемки сайта, в рамках гарантийной поддержки Исполнителем производится однократный перенос разработанного программного обеспечения на аппаратные средства Заказчика. Соответствие программно-аппаратной платформы требованиям настоящего документа обеспечивает Заказчик.

Перед осуществлением переноса Заказчик обеспечивает Исполнителю удаленный доступ к веб-серверу и доступ к базе данных сайта.

\subsection{Дополнительные требования}
