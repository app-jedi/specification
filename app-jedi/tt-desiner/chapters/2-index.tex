\section{Требования к выполнению работы}

\subsection{Файловая структура}

\dirtree{%
  .1 /.
    .2 assets/.
      .3 fonts/\DTcomment{Директория со всеми шрифтами}.
        .4 name-font-0/\DTcomment{Директория для шрифта с названием \texttt{name-font-0}}.
        .4 name-font-1/\DTcomment{Директория для шрифта с названием \texttt{name-font-1}}.
      .3 imgs/\DTcomment{Директория для изображений и фотографий}.
      .3 icons/\DTcomment{Директория для иконок и логотипа}.
      .3 videos/\DTcomment{Директория для видео}.
      .3 gifs/\DTcomment{Директория для гифок}.
    .2 psd/\DTcomment{Директория для макетов}.
    .2 prototypes/\DTcomment{Директория для прототипов}.
}
\subsubsection{Требования к именованию и кодировке файлов }
\begin{enumerate}
  \item Все файлы должны иметь кодировку utf-8;
  \item Файлы должны называться прозрачно и понятно на латиниц.
\end{enumerate}

\subsection{Требования к графическим элементам}
\subsubsection{Требования к изображениям}
\begin{enumerate}
  \item Формат: \texttt{.SVG}
  \item Размер: \texttt{любой}
  \item Расположение: \texttt{./assets/imgs/}
\end{enumerate}

\subsubsection{Требования к логотипу}
\begin{enumerate}
  \item Формат: \texttt{.SVG}
  \item Размер: \texttt{любой}
  \item Расположение: \texttt{./assets/icons/}
  \item Наименование: \texttt{logo.svg}
\end{enumerate}

\subsubsection{Требования к иконкам}
\begin{enumerate}
  \item Формат: \texttt{.SVG}
  \item Расположение: \texttt{./assets/icons/}
\end{enumerate}

\subsubsection{Требования к фоновому изображению и картинкам}
\begin{enumerate}
  \item Формат: \texttt{.PNG}
  \item Размер: \texttt{в соответствии с маштабом максимальным разрешению разрешением}
  \item Расположение: \texttt{./assets/imgs/}
\end{enumerate}

\subsubsection{Требования к шрифтам}
\begin{enumerate}
  \item Формат: \texttt{.ttf} и \texttt{.woff}
  \item Расположение: \texttt{Каждый шрифт должен распологаться в отдельной директории}
  \item Имя директории: \texttt{Директория шрифта должна называться, как сам шрифт}
  \item Расположение директории: \texttt{./assets/fonts/}
\end{enumerate}

\subsubsection{Требования к анимации}
\begin{enumerate}
  \item Описание: Описание анимаций должно быть предоставлено отдельным текстовым файлом, где действия описаны своими словами
  \item Расположение: \texttt{./assets/gifs} и \texttt{./assets/videos}
\end{enumerate}

\subsection{Требования к предоставлени прототипов}
Сторона предоставляющая прототипы: \texttt{Дизайнер (Исполнитель)}.\\
Требования:
\begin{enumerate}
  \item На каждую страницу необходимо разработать по 1 прототипу;
  \item Прототипы должны быть предоставлены в формате \texttt{.PNG};
  \item Прототипы должны располагаться в директории \texttt{prototypes}, находящейся в корневой директории проекта.
\end{enumerate}

\subsection{Требования к предоставления макета }
\begin{enumerate}
  \item Макет должен быть выполнен на сетке, состоящей из 16 колонок
  \item Макет должен быть разработан под 2 типа устройств:\\
  \begin{enumerate}
    \item десктоп
    \item мобильное устройство
  \end{enumerate}
  \item Макет должен быть загружен в онлайн-сервис \texttt{zeplin.com}
  \item Макет должен иметь расширение \texttt{.PSD}
  \item Макет должен быть четко структурирован:
  \begin{enumerate}
    \item Слои не должны быть разбросаны
    \item Слои должны иметь прозрачные и понятные названия
    \item Слои должны быть логически разделены (пример: нельзя объединять текст и заголовок в один слой)
  \end{enumerate}
\end{enumerate}
