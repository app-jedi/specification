\section{Требование к дизайн-концепции}
Для разработки предоставлены:
\begin{enumerate}
  \item Название;
  \item Логотип;
  \item Тексты и описание страниц;
\end{enumerate}

\subsection{Требования к дизайну}
  Необходимо предоставить уникальный дизайн. Точная концепция неутверждена.
  При разработке дизайна сайта необходимо учесть эстетические предпочтения клиента:
\subsubsection{Положительные примеры:}
\begin{enumerate}
  \item www.behance.net/gallery/37814103/Skillful
  \begin{enumerate}
    \item Интересные цвета
    \item Хороший интерфейс
  \end{enumerate}
  \item www.behance.net/gallery/67599963/Argyle-ICO-Landing-page
  \begin{enumerate}
    \item Классные анимации
    \item Интересные цвета
  \end{enumerate}

  \item www.behance.net/gallery/68199171/Sonikpass-Redacted-Identity
  \begin{enumerate}
    \item Классная футуристичная анимация
    \item Интересная работа с типографикой (в частности с размерами заголовков и тестов)
  \end{enumerate}

  \item www.behance.net/gallery/59752143/Design-Trends-2018
  \begin{enumerate}
    \item Классный футуристичный дизайн
    \item Цвета очень хорошие (неплохо было бы использовать на сайте такие же)
  \end{enumerate}

  \item www.sponge.com.ua/services/6
  \begin{enumerate}
    \item хороший дизайн
    \item классная графика
    \item хорошая анимация
  \end{enumerate}
\end{enumerate}

\subsubsection{Отрицательные примеры:}
\begin{enumerate}
  \item dowebdomobile.ru/startups
  \begin{enumerate}
    \item Плохое меню и навигация
    \item НО неплохие внутренние страницы услуг
  \end{enumerate}

  \item magora-systems.ru/
  \begin{enumerate}
    \item уныло
    \item тривиально
    \item шаблонно
  \end{enumerate}
\end{enumerate}
% \begin{enumerate}
%   % \item \href{1}{2}
% \end{enumerate}

\subsection{Порядок утверждения дизайн-концепции}
Под дизайн-концепцией подразумевается вариант графического оформления всех страниц сайта, включая страницы административной панели. Дизайн-концепция представляется в виде набора растровых файлов, либо набора макетов в формате \texttt{.psd}.

Если представленная Исполнителем дизайн-концепция удовлетворяет Заказчика, он должен утвердить ее в течение пяти рабочих дней с момента представления. При этом он может направить Исполнителю список частных доработок, не затрагивающих общую структуру страниц и их стилевое решение. Указанные доработки производятся параллельно с разработкой программных модулей сайта. Внесение изменений в дизайн-концепцию после ее приемки допускается только по дополнительному соглашению сторон.

Если представленная концепция не удовлетворяет требованиям Заказчика, последний предоставляет мотивированный отказ от принятия концепции с указанием деталей, послуживших препятствием для принятия концепции и более четкой формулировкой требований. В этом случае Исполнитель разрабатывает второй вариант дизайн-концепции (дорабатывает, вносит изменения). Обязательства по разработке второго варианта дизайн-концепции Исполнитель принимает только после согласования и подписания дополнительного соглашения о продлении этапа разработки дизайн-концепции. В соглашении должны быть указаны сроки продления этапа разработки дизайн-концепции.

Дополнительные варианты дизайн-концепции разрабатываются Исполнителем отдельно на основании дополнительных соглашений.
