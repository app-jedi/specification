\documentclass[a4paper, fontsize=12bp]{article}
\usepackage{scrextend}
\usepackage[T2A]{fontenc}
\usepackage[utf8]{inputenc}
\usepackage[english,russian]{babel}
\usepackage{amsmath}

\usepackage{parselines}
\linespread{1.5}
\setlength{\parskip}{0.25cm}

\usepackage{graphicx}
\usepackage{float}
\graphicspath{ {./assets/} }

\usepackage{lastpage}
\usepackage[left=2.5cm,right=2.5cm,top=2cm,bottom=2cm,bindingoffset=0cm]{geometry}

\usepackage{enumitem}
\newcounter{enum}

\renewcommand{\labelenumii}{\arabic{enumi}.\arabic{enumii}}


\begin{document}
\begin{center}

\textbf{ДОГОВОР №0000 от <<1>> января 2018 года.}
\end{center}

ООО <<Компания>> в лице Иванова Ивана Ивановича, действующего на основании Устава, именуемого в дальнейшем <<Заказчик>>, с одной стороны, и ИП <<Подрядчик>>, в лице генерального директора Сидорова Сидора Сидоровича, действующего на основании Устава, именуемого в дальнейшем <<Подрядчик>>, с другой стороны, а вместе далее именуемые <<Стороны>>, заключили настоящий договор (далее - Договор) о нижеследующем:


\begin{center}
\textbf{1. Предмет договора}
\end{center}

\stepcounter{enum}
\begin{enumerate}[label=\arabic{enum}.\arabic*]
\item Подрядчик обязуется выполнять по заданию Заказчика определенные работы, а Заказчик обязуется принимать результат работ и оплачивать его.
\end{enumerate}


\begin{center}
\textbf{2. Порядок исполнения договора}
\end{center}

\noindent
\textit{Согласование объема работ}
\stepcounter{enum}
\begin{enumerate}[label=\arabic{enum}.\arabic*.]
\item Объем, срок и стоимость работ согласуются Сторонами в Приложениях к Договору.

\item Обязательства Cторон по проведению и оплате работ возникают после подписания соответствующего Приложения к Договору.

\item Работы стоимостью менее СУММА (РАСШИФРОВКА) рублей могут выполняться без подписания Приложений. Оплата Заказчиком счета Подрядчика, в котором содержится перечень работ, является безусловным согласием на выполнение этих работ.
\end{enumerate}

\noindent 
\textit{Запросы}
\begin{enumerate}[label=\arabic{enum}.\arabic*., resume]
\setcounter{enumii}{3}
\item Заказчик вправе в любое время направить письменный запрос о ходе выполнения работ. Подрядчик обязан предоставить ответ на запрос в срок не позднее 5 (пяти) рабочих дней с момента его получения.
\item Подрядчик вправе в любое время запрашивать у Заказчика сведения и материалы, необходимые для надлежащего выполнения принятых на себя обязательств (далее – Материалы).

\item Заказчик обязуется предоставить Материалы в течение КОЛ-ВО ДНЕЙ (РАСШИФРОВКА) рабочих дней с момента получения запроса Подрядчика. Заказчик гарантирует достоверность, полноту, непротиворечивость и законность Материалов, а также наличие прав на Материалы. Обязательство Заказчика считаются исполненными в случае предоставления Материалов в запрашиваемом объеме и требуемом качестве.
\end{enumerate}

\noindent 
\textit{Ответственные лица}
\begin{enumerate}[label=\arabic{enum}.\arabic*., resume]

\item Стороны назначают своих представителей для решения всех вопросов в связи с выполнением работ по Договору (далее - Ответственные лица). Каждая из Сторон назначает в качестве Ответственного лица одного штатного сотрудника. Перечень Ответственных лиц с указанием должности и контактных данных согласуется в Приложениях.

\item Ответственные лица обладают всеми необходимыми полномочиями для осуществления любых действий, необходимых для выполнения обязательств по Договору, в том числе, но не ограничиваясь: составление и утверждение технического задания и иных требований к результату работ; утверждение бюджета; ведение переписки; сбор и предоставление Материалов; согласование рабочих вопросов; сдача, приемка, утверждение промежуточных и итоговых результатов работ; оформление и подписание актов приема-передачи выполненных работ.

\item При замене Ответственного лица Сторона обязана уведомить об этом другую Сторону не позднее, чем за 2 (два) рабочих дня до даты начала осуществления полномочий новым Ответственным лицом. Уведомление о смене ответственного лица должно содержать ФИО нового Ответственного лица, его должность, телефон и адрес электронной почты (e-mail).

\item Лица, не назначенные Сторонами в качестве Ответственных, не вправе вмешиваться в процесс согласования и приемки работ по Договору. Не допускается одностороннее увеличение Стороной количества Ответственных лиц и назначение в качестве Ответственных лиц внешних консультантов.
\end{enumerate}

\noindent 
\textit{Сроки}
\begin{enumerate}[label=\arabic{enum}.\arabic*., resume]

\item Подрядчик вправе сдать работы по Приложению досрочно, при этом все сроки по Договору переносятся.

\item При несвоевременном и/или ненадлежащем выполнении Заказчиком принятых на себя обязательств (в т.ч. в части оплаты, предоставления Материалов, приемки работ), сроки выполнения работ по Договору автоматически увеличиваются на:
\begin{itemize}
	\item срок задержки выполнения Заказчиком своих обязательств и
	\item срок, необходимый для возобновления выполнения работ Подрядчиком, который может составить не более 10 (десяти) рабочих дней.
\end{itemize}

\item Сроки исполнения обязательств Подрядчика и Заказчика могут быть перенесены на время отсутствия (отпуска) ответственного лица с одной из Сторон. На время согласованного отсутствия одного из ответственных лиц все работы по Приложению приостанавливаются и возобновляются в течение 5 (пяти) рабочих дней с момента выхода на работу ответственного лица.
\end{enumerate}



\begin{center}
\textbf{3. Расчеты сторон}
\end{center}
\stepcounter{enum}
\begin{enumerate}[label=\arabic{enum}.\arabic*.]
\item Стоимость работ по Договору устанавливается в Приложениях.

\item Стоимость работ не облагается НДС в связи с применением Подрядчиком упрощенной системы налогообложения (26,2 НК РФ).

\item Расчеты по Договору производятся Заказчиком в рублях путем перечисления денежных средств на расчетный счет Подрядчика.

\item Обязательства по оплате считаются исполненными с момента зачисления денежных средств на корреспондентский счет банка Подрядчика.

\item Плательщиками по Договору могут выступать третьи лица.

\item Если иное не предусмотрено в Приложении к Договору, работы выполняются на основе 100\% предоплаты. Подрядчик вправе не приступать к выполнению работ до получения предоплаты.

\item Изменение стоимости работ и условий оплаты возможно только по общему согласию Сторон, выраженном в соответствующем Дополнительном соглашении.
\end{enumerate}


\begin{center}
\newpage
\textbf{4. Сдача-приемка работ}
\end{center}
\stepcounter{enum}
\begin{enumerate}[label=\arabic{enum}.\arabic*.]
\item По окончанию работ по Приложению либо счету Подрядчик направляет Заказчику Акт сдачи-приемки выполненных работ (далее – Акт).

\item В течение 5 (пяти) дней с даты получения Акта, Заказчик обязуется принять результат работ путем подписания Акта и передачи его Подрядчику, или отказаться от приемки результата работ, сообщив Подрядчику о необходимости доработки.

\item Под доработкой Стороны понимают приведение результата работ в соответствие с требованиями, согласованными Сторонами в Приложении, в частности, в техническом задании.

\item Если по истечении установленного срока для приемки работ Подрядчик не получит подписанный Акт или мотивированный отказ, работы считаются выполненными надлежащим образом и принятыми Заказчиком в полном объеме.

\item Если Приложением предусмотрено несколько этапов выполнения работ,
Стороны оформляют промежуточный/ые Акт/ы по окончании этапа по правилам настоящего раздела.

\item Переработка принятых работ по Договору осуществляется только при полной оплате Заказчиком повторных работ на основании отдельного Приложения. Стоимость работ определяется Подрядчиком.
\end{enumerate}


\begin{center}
\textbf{5. Права на результат работ}
\end{center}
\stepcounter{enum}
\begin{enumerate}[label=\arabic{enum}.\arabic*.]
\item Если при выполнении работ по Договору Подрядчиком будут созданы результаты интеллектуальной деятельности (далее – РИД), Подрядчик обязуется передать Заказчику исключительное право на РИД в полном объеме. Права передаются без ограничения территории и срока действия. Вознаграждение за отчуждение исключительного права на РИД включено в стоимость работ по Приложению и составляет 10\% от стоимости работ.

\item Право на РИД считается переданным после наступления последнего из следующих событий:
\begin{itemize}
	\item Сторонами подписан Акт по Приложению, в рамках которого был создан РИД.
	\item Заказчик оплатил стоимость работ по Приложению в полном объеме.
\end{itemize}

\item После получения прав на РИД Заказчик самостоятельно предпринимает меры по их дальнейшей защите.

\item Подрядчик гарантирует, что факт передачи РИД не нарушает прав третьих лиц и на момент передачи не существует обстоятельств, дающих возможность третьим лицам предъявить к Заказчику претензии в отношении РИД (за исключением претензий к Материалам).

\item Работники Подрядчика и иные физические лица, участвовавшие в выполнении работ по поручению Подрядчика, имеют право называться автором РИД. Никакое другое лицо, включая Заказчика, не может называться автором РИД. При использовании РИД Заказчик имеет право не указывать его авторов.

\item Заказчик обязуется в нижней части страниц сайтов, разработанных в рамках Договора (далее - Сайты), разместить:
\begin{itemize}
	\item логотип Подрядчика;
	\item гипертекстовую ссылку на сайт Подрядчика (http://www.site.ru/). Ссылка должна быть доступна для индексирования поисковыми системами. Внешний вид и текст ссылки согласуется Сторонами на этапе разработки дизайна Сайта.
\end{itemize}

\item Подрядчик вправе в любое время потребовать снять свой логотип и гипертекстовую ссылку с Сайта. Заказчик обязан удовлетворить требование Подрядчика в течение 5 (пяти) рабочих дней с даты получения соответствующего требования Подрядчика.

\item Заказчик предоставляет Подрядчику право на использование своего имени (наименования), логотипов, товарных знаков, коммерческих обозначений в портфолио и информационных материалах Подрядчика. Заказчик предоставляет Подрядчику право на анонсирование результатов всех работ по Договору.
\end{enumerate}


\begin{center}
\newpage
\textbf{6. Ответственность сторон}
\end{center}
\stepcounter{enum}
\begin{enumerate}[label=\arabic{enum}.\arabic*.]
\item Стороны несут ответственность за неисполнение или ненадлежащее исполнение условий Договора в порядке, предусмотренном Договором и действующим законодательством Российской Федерации.

\item Заказчик несет полную ответственность за содержание, достоверность и законность распространения предоставленных Подрядчику Материалов. Все претензии в отношении Материалов со стороны третьих лиц, в том числе авторов, их наследников, правообладателей, должны быть урегулированы Заказчиком своими силами и за свой счет.

\item В случае возникновения у Подрядчика ущерба, вызванного нарушением Заказчиком законодательства или прав третьих лиц, Заказчик обязан возместить его в полном объеме.

\item Ответственность Сторон по Договору ограничена возмещением документально подтвержденного реального ущерба.

\item За вмешательство Заказчиком в процесс согласования и сдачи-приемки работ по Договору лица, не согласованного в качестве Ответственного лица, Подрядчик вправе потребовать от Заказчика выплатить штраф в размере 10\% (десять процентов) от суммы Договора, а также либо приостановить работы по Договору до прекращения вмешательства данного лица в процесс разработки, либо отказаться от договора в одностороннем порядке.

\item По Договору законные проценты (ст. 317.1 ГК РФ) не начисляются.

\item Стороны освобождаются от ответственности полностью или частично в случае, если в порядке, установленном действующим законодательством, докажут, что причиной неисполнения обязательств явились форс-мажорные обстоятельства, при условии, что они непосредственно влияют на выполнение обязательств по настоящему Договору, а также принятия государственными органами законодательных актов, препятствующих выполнению условий настоящего Договора. В этом случае выполнение обязательств по настоящему Договору откладывается на время действия обстоятельств непреодолимой силы и их
последствий. При наступлении вышеуказанных обстоятельств, каждая из Сторон должна уведомить другую Сторону в письменном виде в течение 5-и дней с момента наступления этих обстоятельств. В случае действия форс-мажорных обстоятельств более 2 (двух) месяцев каждая из Сторон вправе отказаться от Договора в одностороннем порядке.
\end{enumerate}


\begin{center}
\textbf{7. Срок действия и порядок расторжения Договора}
\end{center}
\stepcounter{enum}
\begin{enumerate}[label=\arabic{enum}.\arabic*.]
\item Договор вступает в силу с момента его подписания Сторонами и действует до конца текущего календарного года, а в части расчетов и условий Приложений – до полного выполнения Сторонами принятых на себя обязательств.

\item В случае, если ни одна из Сторон не изъявит в письменной форме желание прекратить сотрудничество по Договору, не менее чем за 30 (тридцать) дней до истечения его срока действия, Договор считается каждый раз автоматически пролонгированным на 1 (один) календарный год.

\item Договор может быть досрочно расторгнут по соглашению Сторон.

\item Заказчик вправе отказаться от исполнения Договора в одностороннем порядке, уведомив об этом Подрядчика не позднее, чем за 7 (семь) рабочих дней до даты расторжения Договора.

\item Подрядчик вправе отказаться от исполнения Договора в одностороннем порядке без возмещения каких-либо убытков, письменно уведомив об этом Заказчика за 7 (семь) календарных дней до даты расторжения Договора, в следующих случаях:
\begin{itemize}
	\item Нарушение Заказчиком сроков оплаты по Приложению более, чем на 10 (десять) календарных дней;

	\item Нарушение Заказчиком сроков предоставления Материалов более, чем на 10 (десять) рабочих дней с момента получения запроса;

	\item Вмешательство со стороны Заказчика лица/лиц, не согласованных в качестве Ответственных.
\end{itemize}

\item При досрочном расторжении Договора Стороны производят взаиморасчеты. Подрядчик направляет Заказчику акт сверки с указанием выполненных, но не закрытых Актами работ по Приложениям. Акт сверки согласуется по правилам, предусмотренным для подписания Актов.
\end{enumerate}


\begin{center}
\newpage
\textbf{8. Порядок решения споров}
\end{center}
\stepcounter{enum}
\begin{enumerate}[label=\arabic{enum}.\arabic*.]
\item Все споры между сторонами решаются путем переговоров на принципах доброй воли. Претензионный порядок решения споров является обязательным. Срок ответа на претензию составляет 30 рабочих дней.

\item В случае невозможности достижения согласия путем переговоров, споры решаются в судебном порядке в Арбитражном суде г. Москвы.
\end{enumerate}


\begin{center}
\textbf{9. Обмен информацией и документами}
\end{center}
\stepcounter{enum}
\begin{enumerate}[label=\arabic{enum}.\arabic*.]
\item Стороны признают надлежащим подписание Договора, приложений, актов, дополнительных соглашений к нему путем обмена отсканированными копиями по электронной почте. Такие документы считаются подписанными простой электронной подписью и приравниваются к документам на бумажном носителе.

\item Оригиналы документов должны быть направлены заказным письмом по почте, курьером или вручены лично не позднее 10 рабочих дней после отправления соответствующих документов по электронной почте. Все документы, направляемые по электронной почте, имеют юридическую силу до момента получения Сторонами их подлинников.

\item Стороны признают надлежащим согласование всех текущих рабочих вопросов в связи с исполнением Договора по электронной почте или в программе Скайп, в том числе направление писем, запросов, других сообщений, проведение звонков и видео-конференций. При отсутствии доказательств фальсификации такая переписка и (или) записи разговоров признаются юридически значимыми и являются надлежащими доказательствами при судебном споре.

\item Для обмена документами и сообщениями по электронной почте должны использоваться адреса Сторон, согласованные в Договоре и Приложениях. Письмо признается надлежащим, если оно направлено одновременно на два адреса получающей Стороны:
\begin{itemize}
	\item адрес электронной почты Стороны, указанный в реквизитах Договора.

	\item адрес электронной почты Ответственного лица Стороны, указанный в Приложении.
\end{itemize}

\item Для обмена сообщениями, проведения звонков и видео-конференций в программе Скайп должны использоваться аккаунты Сторон, согласованные в Договоре и Приложениях.
\end{enumerate}


\begin{center}
\textbf{10. Заключительные положения}
\end{center}
\stepcounter{enum}
\begin{enumerate}[label=\arabic{enum}.\arabic*.]
\item Все Приложения и Дополнительные соглашения к Договору являются его неотъемлемыми частями с момента их надлежащего оформления и подписания обеими Сторонами.

\item Подрядчик вправе привлекать к исполнению Договора третьих лиц без согласия Заказчика.

\item Стороны признают любую информацию, касающуюся заключения и содержания Договора, включая любые приложения и дополнения к нему, коммерческой тайной. Стороны обязуются сохранять конфиденциальный характер такой информации, не разглашая ее третьим лицам без предварительного письменного согласия другой Стороны, за исключением случаев, когда такое раскрытие согласовано Сторонами в Договоре или необходимо для целей исполнения Договора. Указанное положение не относится к общеизвестной или общедоступной информации, а также к случаям раскрытия информации по запросу уполномоченных государственных органов.

\item В рамках Договора Сторонам становятся известны сведения о сотрудниках. Стороны обязуются не использовать данные сведения для того, чтобы пытаться осуществить найм специалистов другой Стороны, прямо привлеченных к выполнению работ) в течение срока действия Договора, а также в течение 2 (двух) лет после его прекращения. В противном случае пострадавшая Сторона имеет право потребовать от Стороны, допустившей данное нарушение, уплату штрафа. Сумма штрафа обсуждается сторонами отдельно, но не может быть менее 25\% от суммы Договора и всех Приложений к нему.

\item Договор составлен в 2-х экземплярах — по одному для каждой из Сторон. Оба экземпляра имеют одинаковую юридическую силу.
\end{enumerate}

\newpage
\begin{center}
\textbf{11. Реквизиты сторон}
\vspace{\baselineskip}
\end{center}

\noindent
\begin{minipage}{0.5\textwidth}
\begin{flushleft}
\begin{center}
\textbf{Подрядчик:}\\
\end{center}
ИП <<Подрядчик>> \\
Адрес:\\
Тел.\\
ОРГН\\
ИНН / КПП\\
Р/сч. №\\
Корр/счет\\
БИК\\
Банк:\\
\vspace{\baselineskip}
Директор:\\
Телефон:\\
E-mail:\\
\vspace{\baselineskip}
Руководитель со стороны Подрядчика:\\
Телефон:\\
E-mail:\\
\end{flushleft}
\end{minipage}
\begin{minipage}{0.5\textwidth}
\begin{flushleft}
\begin{center}
\textbf{Заказчик:}\\
\end{center}
ООО <<Заказчик>>\\
Адрес:\\
Тел.\\
ОРГН\\
ИНН / КПП\\
Р/сч. №\\
Корр/счет\\
БИК\\
Банк:\\
\vspace{\baselineskip}
Директор:\\
Телефон:\\
E-mail:\\
\vspace{\baselineskip}
Руководитель со стороны Заказчика:\\
Телефон:\\
E-mail:\\

\end{flushleft}
\end{minipage}

\begin{center}
\vspace{\baselineskip}
\textbf{12. Подписи сторон}
\vspace{\baselineskip}
\end{center}
\begin{minipage}{\textwidth}
\begin{minipage}{0.5\textwidth}
\begin{center}
Подрядчик $\underset{\text{(подпись) М.П.}}{\underline{\hspace{0.3\textwidth}}}$ Иванов И.И.
\end{center}
\end{minipage}
\begin{minipage}{0.5\textwidth}
\begin{center}
Заказчик $\underset{\text{(подпись) М.П.}}{\underline{\hspace{0.3\textwidth}}}$ Иванов И.И.
\end{center}
\end{minipage}
\end{minipage}
\end{document}
