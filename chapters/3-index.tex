\section{Требование к графическому дизайну сайту}
\subsection{Требования к дизайну}
Разделы «Учебные материалы» и «Генератор вариантов» сайта mathperfect.ru должны быть разработаны в соответствии с имеющимися цветовыми решениями и дизайн-концепцией.

\subsection{Порядок утверждения дизайн-концепции}
Под дизайн-концепцией подразумевается вариант графического оформления всех страниц разделов «Учебные материалы» и «Генератор вариантов», включая страницы административной панели. Дизайн-концепция представляется в виде набора растровых файлов , либо набора макетов в формате .psd.

Если представленная Исполнителем дизайн-концепция удовлетворяет Заказчика, он должен утвердить ее в течение пяти рабочих дней с момента представления. При этом он может направить Исполнителю список частных доработок, не затрагивающих общую структуру страниц и их стилевое решение. Указанные доработки производятся параллельно с разработкой программных модулей сайта. Внесение изменений в дизайн-концепцию после ее приемки допускается только по дополнительному соглашению сторон.

Если представленная концепция не удовлетворяет требованиям Заказчика, последний предоставляет мотивированный отказ от принятия концепции с указанием деталей, послуживших препятствием для принятия концепции и более четкой формулировкой требований. В этом случае Исполнитель разрабатывает второй вариант дизайн-концепции (дорабатывает, вносит изменения). Обязательства по разработке второго варианта дизайн-концепции Исполнитель принимает только после согласования и подписания дополнительного соглашения о продлении этапа разработки дизайн-концепции. В соглашении должны быть указаны сроки продления этапа разработки дизайн-концепции.

Дополнительные варианты дизайн-концепции разрабатываются Исполнителем отдельно на основании дополнительных соглашений.
