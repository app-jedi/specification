\section{Функциональные требования}
\subsection{Классы пользователей}
В системе существуют два вида пользователей:
\begin{enumerate}
  \item Неавторизованные (далее гость)
  \item Авторизованный (далее администратор)
\end{enumerate}

\subsubsection{Права неавторизованного пользователя}
\begin{enumerate}
  \item ОГЭ
  \begin{itemize}
    \item Просмотр
    \item Печать
  \end{itemize}

  \item ЕГЭ профильного уровня
  \begin{itemize}
    \item Просмотр
    \item Печать
  \end{itemize}

  \item ЕГЭ базового уровня
  \begin{itemize}
    \item Просмотр
    \item Печать
  \end{itemize}

  \item Высшая математика
  \begin{itemize}
    \item Просмотр
    \item Печать
  \end{itemize}

  \item Презентации
  \begin{itemize}
    \item Просмотр
    \item Скачивание pptx
    \item Скачивание pdf
  \end{itemize}

  \item Пособие
  \begin{itemize}
    \item Просмотр
    \item Скачивание
  \end{itemize}

  \item Генератор варианта
  \begin{itemize}
    \item Просмотр
    \item Генерирование варианта
    \item Печать
  \end{itemize}
\end{enumerate}

\subsubsection{Права авторизованного пользователя}
Администратор обладает всеми правами Гостя, плюс:
\begin{enumerate}
  \item Список задач
  \begin{itemize}
    \item Просмотр
    \item Удаление
    \item Создание
    \item Редактирование
    \item Управление состоянием
    \item Импорт задач
  \end{itemize}

  \item Список разделов
  \begin{itemize}
    \item Просмотр
    \item Удаление
    \item Создание
    \item Редактирование
    \item Управление состоянием
  \end{itemize}

  \item Список презентаций
  \begin{itemize}
    \item Просмотр
    \item Удаление
    \item Создание
    \item Редактирование
    \item Управление состоянием
  \end{itemize}

  \item Список пособий
  \begin{itemize}
    \item Просмотр
    \item Удаление разделов
    \item Создание разделов
    \item Редактирование разделов
    \item Управление состоянием
  \end{itemize}

  \item Редактирование/создание задачи
  \begin{itemize}
    \item Просмотр
    \item Редактирование
  \end{itemize}

  \item Редактирование/создание раздела
  \begin{itemize}
    \item Просмотр
    \item Редактирование
  \end{itemize}

  \item Редактирование/создание презентации
  \begin{itemize}
    \item Просмотр
    \item Редактирование
  \end{itemize}

  \item Редактирование/создание пособия
  \begin{itemize}
    \item Просмотр
    \item Редактирование
  \end{itemize}

\end{enumerate}

\subsection{Требования к представлению сайта}
  \paragraph{Учебные материалы} -- входная точка проекта. Основной контент данной страниц -- 6 блоков:
  \begin{enumerate}
    \item ЕГЭ базового уровня
    \item ЕГЭ профильного уровня
    \item ОГЭ
    \item Высшая математика
    \item Презентации
    \item Пособия
  \end{enumerate}

При клике 1-4 происходит переход на страницу -- список типов задач (разделов). Данный список представляет из себя типы задач по соответствующим темам. При клике на один из разделов происходит переход на станицу со списком задач соответствующих данной теме.

При кликах на 5-6 происходит переход к списку презентаций и пособий с ссылками для скачивания.

\paragraph{Генератор вариантов}
Из разделов 1-3 и их подразделов необходимо генерировать случаный вариант с заданной сложностью.
Алгоритм:

\subsection{Требования к системе управления сайтом}
В системе управления сайта (далее админке) возможно:
\begin{enumerate}
  \item Создание/редактирование задач
  \item Имопорт задач
  \item Создание/редактирование разделов
  \item Создание/редактирование презентаций
  \item Создание/редактирование Пособий
\end{enumerate}
