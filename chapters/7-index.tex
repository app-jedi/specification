\section{Требования к видам обеспечения}
\subsection{Требования к информационному обеспечению}
\subsubsection{Требования к хранению данных}
Данные делятся на два типа: внутрение и внешние.
Внутрение:
\begin{itemize}
    \item Задачи
    \item Разделы
\end{itemize}
Внешние:
\begin{itemize}
    \item Пособия
    \item Презентации
\end{itemize}

Изображения, которые относятся к шаблону сайта считаются внутренними данными. Остальные изобржание рекомендуется относить к внешним данным и хранить на стороннем сервере.
\subsubsection{Требования к языкам программирования}
Проект делится на две части:
\begin{itemize}
    \item front-end
    \item back-end
\end{itemize}
Стэк технологий для front-end:
\begin{enumerate}
    \item HTML 5
    \item CSS (SCSS)
    \item JavaScript (ES6, ES7, ES8)
    \item React
\end{enumerate}
Возможно добавление вспомогательных framework'ов для ускорения и улучшения качества разработки, таких как:
\begin{enumerate}
    \item React.Route
    \item Ramda
    \item и так далее...
\end{enumerate}
\subsubsection{Требования к изображениям}
Рекомендуется предоставлять одно изображения в трех размерах:
\begin{enumerate}
    \item Small
    \item Middle
    \item Big
\end{enumerate}
Рекомендуемые разрешения для формата "Small":
\begin{enumerate}
    \item $120 \times 160$
    \item $240 \times 400$
\end{enumerate}
Рекомендуемые разрешения для формата "Middle":
\begin{enumerate}
    \item $ 360 \times 480 $
    \item $ 360 \times 640 $
\end{enumerate}
Рекомендуемые разрешения для формата "Big":
\begin{enumerate}
    \item $ 1080 \times 1920 $
    \item $ 960 \times 1280 $
\end{enumerate}
\subsubsection{Требования к импортируемым .Tex файлам}
TeX файл должен содержать в себе хотя бы один из перечисленных ниже пунктов:
\begin{enumerate}
    \item Условие
    \item Решение
    \item Ответ
\end{enumerate}

Так в файле могут быть заполнены необязательные поля:
\begin{enumerate}
    \item Сложность
    \item Заголовок
    \item ID задачи в системе
    \item Изображение
\end{enumerate}

Все выше перечисленные поля заполняются в соответсвии с шаблоном приведенным в приложении.

\subsection{Требования к программному обеспечению}
\subsubsection{Сервер}
Для выполнения данного ТЗ необходим VPS. Программные/системные требования выдвигаемые к VPS:
\begin{enumerate}
    \item ...
\end{enumerate}
\subsubsection{Клиент}
Поддерживаемые браузеры версии не ниже:
\begin{enumerate}
    \item EI 11
    \item Edge 17
    \item Google Chrome 67
    \item Firefox 60
    \item Safari 11.1
    \item Chrome for Android
    \item iOS Safari 11.3
\end{enumerate}

\subsection{Требования к аппаратному обеспечению}
Аппаратные требования выдвигаемые к VPS:
\begin{enumerate}
    \item ...
\end{enumerate}

\subsection{Требования к лингвистическому обеспечению}
Разделы сайта должны быть выполнены на русском языке.

\subsection{Требования к эргономике и технической эстетике}
Разделы сайта должны корректно отображаться на экранах современных мобильных устройств и персональных компьютеров.
Элементы управления должны иметь единое отображение на всех страницах, а также должны быть сгруппированы по смысловому значению.
Интерфейс административной панели должен быть интуитивно понятным и должен обеспечивать прозрачную работу администратора.
