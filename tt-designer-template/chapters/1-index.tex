\section{Общие положения}

\subsection{Предмет разработки}

\subsection{Эксплуатационное назначение}

\subsection{Функциональное назначение}

\subsection{Целевая аудитория}
\paragraph{Персонаж 1:}
\begin{enumerate}
  \item Пол: м/ж
\end{enumerate}

\subsection{Назначение документа}
В настоящем документе приводится полный набор требований к реализации дизайна сайта.
Подпись Заказчика и Исполнителя на настоящем документе подтверждает их согласие с ниже следующими фактами и условиями:
\begin{enumerate}
  \item Заказчик подготовил и разработал настоящий документ, именуемый Техническое Задание, который содержит перечень требований к выполняемым работам.
  \item Исполнитель согласен со всеми положениями настоящего Технического Задания.
  \item Заказчик вправе требовать от Исполнителя выполнения работ либо оказания услуг, прямо описанных в настоящем Техническом Задании.
  \item Исполнитель обязуется выполнить работы в объёме, указанном в настоящем Техническом Задании.
  \item Заказчик не вправе требовать от Исполнителя соблюдения каких-либо форматов и стандартов, если это не указано в настоящем Техническом Задании.
  \item Все неоднозначности, выявленные в настоящем Техническом Задании после его подписания, подлежат двухстороннему согласованию между Сторонами. В процессе согласования могут быть разработаны дополнительные требования, которые оформляются дополнительным соглашением к Договору и соответствующим образом оцениваются.
\end{enumerate}
