\section{Общие положения}
\subsection{Предмет разработки}
  Предметом разработки являются разделы «Учебные материалы» и «Генератор вариантов» сайта mathperfect.ru, включающие контент в виде базы задач, учебных пособий и презентаций по математике, а также генератор тестовых вариантов с задачами и систему управления контентом соответствующих разделов.

\subsection{Эксплуатационное назначение}
  Эксплуатационным назначением разделов «Учебные материалы» и «Генератор вариантов» является предоставление пользователю структурированной информации для обеспечения возможности самостоятельного изучения математики.

\subsection{Функциональное назначение}
  Функциональным назначением разделов сайта является:
\begin{enumerate}
  \item Предоставление пользователю авторских учебных пособий и презентаций по математике;
  \item Предоставление пользователю базы математических задач разного уровня сложности и тематики;
  \item Предоставление пользователю возможности формировать тестовые варианты для подготовки к единому государственному экзамену (ЕГЭ) базового и профильного уровня, а также основного государственного экзамена (ОГЭ) по математике;
\end{enumerate}

Раздел "Учебные материалы" состоит из 6 подразделов:
\begin{enumerate}
  \item ЕГЭ базового уровня
  \item ЕГЭ профильного уровня
  \item ОГЭ
  \item Учебные пособия
  \item Презентации
  \item Высшая математика
\end{enumerate}
В подразделах «ЕГЭ базового уровня», «ЕГЭ профильного уровня», «ОГЭ» и «Высшая математика» находятся вложенные списки тем (содержания), содержащие соответствующие задачи различного уровня сложности.

В подразделах «Учебные пособия» и «Презентации» содержатся, соответственно, списки авторских учебных пособий и презентаций по математике. Все учебные пособия имеют расширение .pdf. Все презентации имеют расширения .pptx и .pdf.

В разделе «Генератор вариантов» содержится форма для генерации тестового варианта ЕГЭ (базового или профильного уровня), или ОГЭ.

\subsection{Назначение документа}
В настоящем документе приводится полный набор требований к реализации разделов «Учебные пособия» и «Генератор вариантов» сайта mathperfect.ru.
Подпись Заказчика и Исполнителя на настоящем документе подтверждает их согласие с нижеследующими фактами и условиями:
\begin{enumerate}
  \item Исполнитель подготовил и разработал настоящий документ, именуемый Техническое Задание, который содержит перечень требований к выполняемым работам.
  \item Заказчик согласен со всеми положениями настоящего Технического Задания.
  \item Заказчик вправе требовать от Исполнителя выполнения работ либо оказания услуг, прямо описанных в настоящем Техническом Задании.
  \item Исполнитель обязуется выполнить работы в объёме, указанном в настоящем Техническом Задании.
  \item Заказчик не вправе требовать от Исполнителя соблюдения каких-либо форматов и стандартов, если это не указано в настоящем Техническом Задании.
  \item Все неоднозначности, выявленные в настоящем Техническом Задании после его подписания, подлежат двухстороннему согласованию между Сторонами. В процессе согласования могут быть разработаны дополнительные требования, которые оформляются дополнительным соглашением к Договору и соответствующим образом оцениваются.
\end{enumerate}
